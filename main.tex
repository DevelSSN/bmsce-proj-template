% Options for packages loaded elsewhere
\PassOptionsToPackage{unicode}{hyperref}
\PassOptionsToPackage{hyphens}{url}
\PassOptionsToPackage{dvipsnames,svgnames,x11names}{xcolor}
\documentclass[
  12pt,
  a4paper,
]{book}
\usepackage{beamerarticle} % needs to be loaded first
\usepackage{xcolor}
\usepackage[left=1in,right=1in,top=1in,bottom=1in]{geometry}
\usepackage{amsmath,amssymb}
\setcounter{secnumdepth}{3}
\usepackage{iftex}
\ifPDFTeX
  \usepackage[T1]{fontenc}
  \usepackage[utf8]{inputenc}
  \usepackage{textcomp} % provide euro and other symbols
\else % if luatex or xetex
  \usepackage{unicode-math} % this also loads fontspec
  \defaultfontfeatures{Scale=MatchLowercase}
  \defaultfontfeatures[\rmfamily]{Ligatures=TeX,Scale=1}
\fi
\usepackage{lmodern}
\ifPDFTeX\else
  % xetex/luatex font selection
\fi
% Use upquote if available, for straight quotes in verbatim environments
\IfFileExists{upquote.sty}{\usepackage{upquote}}{}
\IfFileExists{microtype.sty}{% use microtype if available
  \usepackage[]{microtype}
  \UseMicrotypeSet[protrusion]{basicmath} % disable protrusion for tt fonts
}{}
\usepackage{setspace}
\makeatletter
\@ifundefined{KOMAClassName}{% if non-KOMA class
  \IfFileExists{parskip.sty}{%
    \usepackage{parskip}
  }{% else
    \setlength{\parindent}{0pt}
    \setlength{\parskip}{6pt plus 2pt minus 1pt}}
}{% if KOMA class
  \KOMAoptions{parskip=half}}
\makeatother
\usepackage{longtable,booktabs,array}
\usepackage{calc} % for calculating minipage widths
% Correct order of tables after \paragraph or \subparagraph
\usepackage{etoolbox}
\makeatletter
\patchcmd\longtable{\par}{\if@noskipsec\mbox{}\fi\par}{}{}
\makeatother
% Allow footnotes in longtable head/foot
\IfFileExists{footnotehyper.sty}{\usepackage{footnotehyper}}{\usepackage{footnote}}
\makesavenoteenv{longtable}
\setlength{\emergencystretch}{3em} % prevent overfull lines
\providecommand{\tightlist}{%
  \setlength{\itemsep}{0pt}\setlength{\parskip}{0pt}}
\usepackage[]{natbib}
\bibliographystyle{plainnat}
\makeatletter
\@ifpackageloaded{subfig}{}{\usepackage{subfig}}
\@ifpackageloaded{caption}{}{\usepackage{caption}}
\captionsetup[subfloat]{margin=0.5em}
\AtBeginDocument{%
\renewcommand*\figurename{Figure}
\renewcommand*\tablename{Table}
}
\AtBeginDocument{%
\renewcommand*\listfigurename{List of Figures}
\renewcommand*\listtablename{List of Tables}
}
\newcounter{pandoccrossref@subfigures@footnote@counter}
\newenvironment{pandoccrossrefsubfigures}{%
\setcounter{pandoccrossref@subfigures@footnote@counter}{0}
\begin{figure}\centering%
\gdef\global@pandoccrossref@subfigures@footnotes{}%
\DeclareRobustCommand{\footnote}[1]{\footnotemark%
\stepcounter{pandoccrossref@subfigures@footnote@counter}%
\ifx\global@pandoccrossref@subfigures@footnotes\empty%
\gdef\global@pandoccrossref@subfigures@footnotes{{##1}}%
\else%
\g@addto@macro\global@pandoccrossref@subfigures@footnotes{, {##1}}%
\fi}}%
{\end{figure}%
\addtocounter{footnote}{-\value{pandoccrossref@subfigures@footnote@counter}}
\@for\f:=\global@pandoccrossref@subfigures@footnotes\do{\stepcounter{footnote}\footnotetext{\f}}%
\gdef\global@pandoccrossref@subfigures@footnotes{}}
\@ifpackageloaded{float}{}{\usepackage{float}}
\floatstyle{ruled}
\@ifundefined{c@chapter}{\newfloat{codelisting}{h}{lop}}{\newfloat{codelisting}{h}{lop}[chapter]}
\floatname{codelisting}{Listing}
\newcommand*\listoflistings{\listof{codelisting}{List of Listings}}
\makeatother
\usepackage{bookmark}
\IfFileExists{xurl.sty}{\usepackage{xurl}}{} % add URL line breaks if available
\urlstyle{same}
\hypersetup{
  pdftitle={Default Title},
  pdfauthor={Author 1; Author 2},
  pdfkeywords={Keyword1, Keyword2, Keyword3},
  colorlinks=true,
  linkcolor={Maroon},
  filecolor={Maroon},
  citecolor={Blue},
  urlcolor={Blue},
  pdfcreator={LaTeX via pandoc}}

\title{Default Title}
\usepackage{etoolbox}
\makeatletter
\providecommand{\subtitle}[1]{% add subtitle to \maketitle
  \apptocmd{\@title}{\par {\large #1 \par}}{}{}
}
\makeatother
\subtitle{Optional Subtitle}
\author{Author 1 \and Author 2}
\date{2025-01-17}

\begin{document}
\frontmatter
\maketitle
\begin{abstract}
This is a brief abstract of the document.
\end{abstract}

\renewcommand*\contentsname{Table of Contents}
{
\hypersetup{linkcolor=blue}
\setcounter{tocdepth}{2}
\tableofcontents
}
\listoffigures
\listoftables
\setstretch{1.5}
\mainmatter
\chapter{GENERAL GUIDELINES:}\label{general-guidelines}

\begin{itemize}
\tightlist
\item
  For each figure, append it with a caption like ``Fig 4.1: One-Line
  Description''
\item
  Again, include a caption for each table as ``Table 1.1: Data of
  population''
\item
  Maximize the referencing of each image into the text of the report.

  \begin{itemize}
  \tightlist
  \item
    However, do this only if it is either necessary, or it enhances the
    clarity of the text.
  \end{itemize}
\end{itemize}

\chapter{Introduction}\label{sec:Intro}

\section{Project Overview}\label{sec:IntroOverview}

\begin{itemize}
\item
  In this introduction, please write relevant information about any
  technologies the project uses, or its features.
\item
  Also, mention the references/bibliographies required here.
\end{itemize}

\section{Project Motivation}\label{sec:IntroMotive}

\begin{itemize}
\tightlist
\item
  Here, mention the motivation behind the project.
\end{itemize}

\begin{enumerate}
\def\labelenumi{\arabic{enumi}.}
\tightlist
\item
  Why the project exists?
\item
  What problem the project intends to solve?
\item
  Any ideas on how the project goes about doing that?
\end{enumerate}

\chapter{Requirements}\label{sec:Requirements}

\begin{itemize}
\item
  List all requirements of the project, like software required/tech
  stack/ technology.
\item
  Also specify the hardware components and tools required.
\item
  Additional requirements are as per the project.
\item
  Each requirement category must have a section dedicated to it.
\end{itemize}

\chapter{Description of the project}\label{sec:Project}

\begin{itemize}
\tightlist
\item
  Explain the working of the project.
\item
  As per department/guide concerned, modify this to be several
  sections/chapters long.
\end{itemize}

\chapter{Output(OR Outcome) of the project}\label{sec:Conclusion}

\begin{itemize}
\tightlist
\item
  Describe the outcome of the project.

  \begin{itemize}
  \tightlist
  \item
    This may include the output of the code, or the pictures of this
    project.
  \item
    This is flexible, as the content can be scattered throughout.
  \item
    Do this as per the direction of the guide.
  \end{itemize}
\end{itemize}

\textbf{THIS CAN ALSO BE THE CONCLUSION OF THE PROJECT, HENCE MODIFY THE
CHAPTER AND ITS SECTIONS ACCORDINGLY}

\chapter{Bibliography}\label{sec:bib}

\begin{itemize}
\tightlist
\item
  Append the bibliography, or the references here.
\item
  Style must be ascertained before insertion.
\end{itemize}

\backmatter
\nocite{ref1, ref2}
\end{document}
